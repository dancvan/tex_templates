
\documentclass[preprint,12pt]{elsarticle}


\usepackage{titlesec}
\usepackage{verbatim}
\usepackage[title,titletoc]{appendix}
\usepackage{booktabs}
\usepackage{graphicx}
\usepackage{setspace}
\usepackage{amssymb}
\usepackage{bigstrut}
\usepackage{lineno}
\usepackage{afterpage}
\usepackage{graphicx}
\usepackage{array}
\usepackage{multirow}
\usepackage{amsfonts,epsfig}
\usepackage{rotating}
\usepackage{algorithmicx}
\usepackage{color}
\usepackage{subcaption}
\captionsetup{compatibility=false}
\usepackage[normalem]{ulem}
\usepackage{pstricks,pst-node,amsmath}
\usepackage{multirow}
\usepackage{graphicx,amsmath}
\usepackage{flushend}
\usepackage{amssymb,amsfonts,amsbsy,epsfig}
\usepackage{lscape}
\usepackage[noend]{algpseudocode}
\usepackage{epstopdf}
\usepackage{amsmath,mathtools}
\usepackage{framed}
\usepackage{algorithm}
\usepackage{hyperref}
\usepackage[margin=0.5 in]{geometry}
\usepackage[symbol]{footmisc}
\usepackage [english]{babel}
\usepackage [autostyle, english = american]{csquotes}
\MakeOuterQuote{"}
\usepackage{amssymb}
\usepackage{xspace}
\usepackage{setspace}
\usepackage{array}
\usepackage{textcomp}


\journal{Saulson 600}

\begin{document}
\sloppy
{\LARGE\textbf{Beamsplitters (Haus)}}
\\
\section{Reciprocity principle}
\label{s:hho}

Figure \ref{fig:CT} shows all phases of HHO, which are described in the next subsections.

%\begin{figure}[http]
%\centering
%\includegraphics[scale=0.6]{CT6}
%\caption{Different phases of HHO}
%\label{fig:CT}
%\end{figure}
%%%%%%%%%%%%%%%%%%%%%%%%%%%%%%

%%%%%%%%%%%%%%%%%%%%%%%%%%%%%%
\subsection{Exploration phase}

In HHO, the Harris' hawks perch randomly on some locations and wait to detect a prey based on two strategies.

\begin{equation} \label{eq:1}
X(t+1) = \left\{\begin{matrix}
X_{rand}(t)-r_{1}\left | X_{rand}(t)-2r_{2}X(t) \right |  &  q\geq 0.5   \\
(X_{rabbit}(t)-X_{m}(t))-r_{3}(LB+r_{4}(UB-LB)) &  q<0.5
\end{matrix}\right.
 \end{equation}

\noindent where $X(t+1)$ is the position vector of hawks in the next iteration $t$, $X_{rabbit}(t)$ is the position of rabbit,  $X(t)$ is the current position vector of hawks, $r_{1}$, $r_{2}$, $r_{3}$, $r_{4}$, and $q$ are random numbers inside (0,1), which are updated in each iteration, $LB$ and $UB$ show the upper and lower bounds of variables, $X_{rand}(t)$ is a randomly selected hawk from the current population, and $X_{m}$ is the average position of the current population of hawks.
The average position of hawks is attained using Eq. (\ref{eq:2}):
\begin{equation} \label{eq:2}
X_{m}(t)=\frac{1}{N}\sum_{i=1}^{N}X_{i}(t)
\end{equation}
where $X_{i}(t)$ indicates the location of each hawk in iteration $t$ and $N$ denotes the total number of hawks.

\subsection{Transition from exploration to exploitation}

To model this step, the energy of a rabbit is modeled as:
\begin{equation} \label{eq:3}
E=2E_{0}(1-\frac{t}{T})
\end{equation}
where $E$ indicates the escaping energy of the prey, $T$ is the maximum number of iterations, and $E_{0}$ is the initial state of its energy.  The time-dependent behavior of $E$ is also demonstrated in Fig. \ref{fig:E}.

%%%%%%%%%%%%%%%%%%%%%%%%%%%%%%

\subsection{Exploitation phase}

\subsubsection{Soft besiege}
This behavior is modeled by the following rules:

\begin{equation} \label{eq:4}
X(t+1)=\Delta X(t)-E\left |JX_{rabbit}(t)-X(t)\right |
\end{equation}
\begin{equation} \label{eq:deltaX}
\Delta X(t)=X_{rabbit}(t)-X(t)
\end{equation}
where $\Delta X(t)$ is the difference between the position vector of the rabbit and the current location in iteration $t$, $r_{5}$ is a random number inside (0,1), and $J=2(1-r_{5})$ represents the random jump strength of the rabbit throughout the escaping procedure. The $J$ value changes randomly in each iteration to simulate the nature of rabbit motions.

\subsubsection{Hard besiege}
In this situation, the current positions are updated using Eq. (\ref{eq:hard}):
\begin{equation} \label{eq:hard}
X(t+1)=X_{rabbit}(t)-E \left |\Delta X(t) \right |
\end{equation}

A simple example of this step with one hawk is depicted in Fig. \ref{fig:movehard}.


%%%%%%%%%%%%%%%%%%%%%%%%%%%%%
%%%%%%%%%%%%%%%%%%%%%%%%%%%%%%

\subsubsection{Soft besiege with progressive rapid dives}

To perform a soft besiege, we supposed that the hawks can evaluate (decide) their next move based on the following rule in Eq. (\ref{eq:6}):
\begin{equation} \label{eq:6}
Y=X_{rabbit}(t)-E\left |JX_{rabbit}(t)-X(t)\right |
\end{equation}

We supposed that they will dive based on the LF-based patterns using the following rule:
\begin{equation} \label{eq:7}
Z=Y+S\times LF(D)
\end{equation}
where $D$ is the dimension of problem and $S$ is a random vector by size $1\times D$ and LF is the levy flight function, which is calculated using Eq. (\ref{eq:lf}):
\begin{equation} \label{eq:lf}
LF(x)=0.01\times \frac{u\times \sigma }{\left | v \right |^{\frac{1}{\beta }}}, \sigma =\left ( \frac{\Gamma (1+\beta )\times sin(\frac{\pi\beta}{2})}{\Gamma(\frac{1+\beta}{2})\times \beta\times2^{(\frac{\beta-1}{2})})}  \right )^{\frac{1}{\beta}}
\end{equation}
where $u$, $v$ are random values inside (0,1), $\beta$ is a default constant set to 1.5.


Hence, the final strategy for updating the positions of hawks in the soft besiege phase can be performed by Eq. (\ref{eq:main}):
\begin{equation} \label{eq:main}
X(t+1)=\left\{\begin{matrix}
Y & if F(Y)<F(X(t)) \\
Z  & if F(Z)<F(X(t)) \\
\end{matrix}\right.
\end{equation}
where $Y$ and $Z$ are obtained using Eqs.(\ref{eq:6}) and (\ref{eq:7}).

A simple illustration of this step for one hawk is demonstrated in Fig. \ref{fig:move}.


\subsubsection{Hard besiege with progressive rapid dives}

The following rule is performed in hard besiege condition:
\begin{equation} \label{eq:mainhard2}
X(t+1)=\left\{\begin{matrix}
Y & if F(Y)<F(X(t)) \\
Z  & if F(Z)<F(X(t)) \\
\end{matrix}\right.
\end{equation}


where $Y$ and $Z$ are obtained using new rules in Eqs.(\ref{eq:newy}) and (\ref{eq:newz}).

\begin{equation} \label{eq:newy}
Y=X_{rabbit}(t)-E\left |JX_{rabbit}(t)-X_{m}(t)\right |
\end{equation}
\begin{equation} \label{eq:newz}
Z=Y+S\times LF(D)
\end{equation}
where $X_{m}(t)$ is obtained using Eq. (\ref{eq:2}).

A simple example of this step is demonstrated in Fig. \ref{fig:movehard2}.


\subsection{Pseudocode of HHO}
The pseudocode of the proposed HHO algorithm is reported in Algorithm \ref{alg:HHO}.

\begin{algorithm}

\caption{Pseudo-code of HHO algorithm}\label{alg:HHO}
\begin{algorithmic}
\State \textbf{Inputs}: The population size $N$ and maximum number of iterations $T$
\State \textbf{Outputs}: The location of rabbit and its fitness value
\State Initialize the random population $X_{i}(i=1,2,\ldots,N)$
\While{(stopping condition is not met)}
\State Calculate the fitness values of hawks
\State Set \textbf{$X_{rabbit}$} as the location of rabbit (best location)
\For{(each hawk ($X_{i}$))}
\State Update the initial energy $E_{0}$ and jump strength $J$ \Comment{\texttt{E$_{0}$=2rand()-1, J=2(1-rand())}}
\State Update the $E$ using Eq. (\ref{eq:3})

\If{ ($|E|\geq 1$)} \Comment{\texttt{Exploration phase}}
\State Update the location vector using Eq. (\ref{eq:1})
\EndIf
\If {($|E|< 1$)} \Comment{\texttt{Exploitation phase}}
\If{($r\geq$0.5 and $|E|\geq 0.5$ ) }  \Comment{Soft besiege}
\State Update the location vector using Eq. (\ref{eq:4})
%\EndIf
\ElsIf{($r\geq$0.5 and $|E|< 0.5$ ) } \Comment{Hard besiege}
\State Update the location vector using Eq. (\ref{eq:hard})
%\EndIf
\ElsIf{($r<$0.5 and $|E|\geq 0.5$ ) } \Comment{Soft besiege with progressive rapid dives}
\State Update the location vector using Eq. (\ref{eq:main})
%\EndIf
\ElsIf{($r<$0.5 and $|E|< 0.5$ ) }  \Comment{Hard besiege with progressive rapid dives}
\State Update the location vector using Eq. (\ref{eq:mainhard2})
\EndIf
\EndIf
\EndFor
\EndWhile
\State \textbf{Return} \textbf{$X_{rabbit}$}
\end{algorithmic}
\end{algorithm}
%%%%%%%%%%%%%%%%%%%%%%%%%%%%%%%%%

\section*{References}

\textbf{Harris Hawks Optimization: Algorithm and Applications, Ali Asghar Heidari and Seyedali Mirjalili and Hossam Faris and Ibrahim Aljarah and Majdi Mafarja and Huiling Chen, Future Generation Computer Systems, 2019}.


\end{document}
